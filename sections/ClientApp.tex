\section{Flutter Development Phase}
\subsection{Use-Case Diagram}
\subsubsection{Global Use-Case Diagram}
In real life, the Use Cases for a small system will be fairly well worked out before work begins
on building the system. Large system tends to work more in parallel than linearly. So, some
people will work on Use Cases, while others start building code for the Use Cases already
completed. Since our Order Processing system is relatively small, we have completed all the
Use Cases for it before building the system and we have detailed some basic Scenarios as
well.
They seem pretty simple.
To fully understand the system's purpose, you must know who the system is for, that is, who
will be using the system. Different user types are represented as actors.
An actor is anything that exchanges data with the system. An actor can be a user, external
hardware, or another system. So, finding actors is one of the first steps in defining our system
use. Each type of external phenomenon with which the system must interact is represented by
an actor. We defined four actors for our system and we determined in the next step, with the
use case diagram, the functionalities that are related to each of them.
\subsubsection{Roles}
This figure allows to draw up the actors within the system, there will be four actors and each
user will have different functionality to the others.

\subsubsection{Global Use-Case Diagram and  Description}
The rest of this chapter is the Use Case diagram for our application’s system.
You’ll see a lot of uses relationships between Use Cases. Uses are found early in the process,
and allow you to show commonality between parts of the system. Extends tends to be added
later, when we find some new requirement or functionality that extends the current system.
Since we haven’t built the first system yet, we don’t have anything to extend.
\subsubsection{Detailed Use-Case Diagrams}
\paragraph{administrator management module}
\paragraph{recruiter  management module}
\paragraph{candidate  management module}
\subsection{Class Diagrams}
\subsubsection{Global Class Description}
\subsubsection{Detailed Class Diagrams}
\paragraph{administrator management module}
\paragraph{recruiter  management module}
\paragraph{candidate  management module}
\paragraph{chatbot module}
\paragraph{live chat module}
\paragraph{Mailing Module}
\subsection{Sequence Diagrams}
\subsubsection{Global Sequence Description}
\subsubsection{Detailed Sequence Diagrams}
\paragraph{administrator management module}
\paragraph{recruiter  management module}
\paragraph{candidate  management module}
\paragraph{chatbot module}
\paragraph{live chat module}
\paragraph{Mailing Module}
\subsection{Mockup}
